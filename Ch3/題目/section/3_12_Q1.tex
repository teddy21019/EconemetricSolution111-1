\begin{frame}
    \frametitle{a) $H_0$ : 在 2000 $sqft$ 時邊際效果$\le $ 13,000 }

    使用二次迴歸模型 
    \begin{equation*}
        PRICE = \alpha_1 + \alpha_2 SQFT^2 + e
    \end{equation*}
    檢定以下假設:

    \begin{quote}
    將 2,000 平方英尺房屋的面積增加 100 平方英尺,對預期房價的邊際影響小於等於13,000 美元,相對於期邊際影響將大於 13,000 美元的對立假設。
    \end{quote}
    使用 $5\%$ 的顯著水準。明確指出使用的檢定統計量,拒絕區域的p值。
\end{frame}

\begin{frame}[plain]
    此模型的預期房價為 
    \begin{equation*}
        E[PRICE] = \alpha_1 + \alpha_2 SQFT^2
    \end{equation*}
    因此面積的邊際效果為
    \begin{equation}
        2\alpha_2 SQFT
    \end{equation}
    \begin{block}{$H_0, H_a$}
        \begin{align*}
            H_0 &: 2\alpha_2 SQFT \le 13 \\
            H_a &: 2\alpha_2 SQFT > 13 \\
        \end{align*}
    \end{block}

    (a) 小題中,$SQFT$ 應帶入 20
\end{frame}

\begin{frame}
    \frametitle{Stata 指令}
    \begin{itemize}
        \item Stata 本身沒有單尾的檢定
        \item 指令 \texttt{test} 做的是雙尾檢定
        \item \texttt{margin} 的值也是雙尾檢定
        \item 需要自己從統計量上下手
    \end{itemize}

    對立假設是 $H_a : 2\alpha_2 SQFT > 13$ ,因此要進行右尾檢定
\end{frame}

\begin{frame}[plain]
    \begin{enumerate}
        \item 算出在 2000 平方英尺($SQFT=20$)時的邊際效果 $\hat{m}$
        \item 找出 t 統計量
        \footnote{如何找出 $\hat{m}$以及$s.e(m)$?}
        \begin{equation*}
            t = \frac{\hat{m} - 13}{s.e(m)}
        \end{equation*}
        \item 計算p-value,也就是右尾的面積
        \begin{alertblock}{注意使用的語法}
            Stata 計算 t分佈右尾的「函數指令」為\texttt{ttail df t},但是不能只打這樣,要加上 \texttt{di} 才會計算!
        \end{alertblock}
        \item 判斷虛無假設
    \end{enumerate}
\end{frame}