\begin{frame}
    \frametitle{化工廠的例子}

    想要知道銷售與
    \begin{itemize}
        \item 資本
        \item 勞動
        \item 原料
    \end{itemize}
    之間的關係

    \begin{align*}
        \ln(SALES_{it}) = \beta_1 &+ \beta_2 \ln(CAPITAL_{it})
        + \beta_3 \ln(LABOR_{it}) \\
        & + \beta_4 \ln(MATERIAL_{it}) 
        + u_i + e_{it}
    \end{align*}

\end{frame}

\begin{frame}[fragile]
    \frametitle{OLS}

    不管廠商之間生產力的差別,統一做回歸
    \begin{lstlisting}
eststo est_a_ols : reg lsales lcapital llabor lmaterials \end{lstlisting}
    

\end{frame}

\begin{frame}[fragile]
    \frametitle{穩健標準差}
    \begin{itemize}
        \item 
    (CH8) 如果 $e_{it}$ 的變異數隨個體而異,可以請 stata 考慮進去,回報「穩健標準誤差」
    \begin{lstlisting}
eststo est_a_r : reg lsales lcapital llabor lmaterials, r \end{lstlisting}
        \item  
        如果想讓同一群廠商變異數一樣,可以指定為「群聚穩健標準誤差(clustered robustness SE)」
        \begin{lstlisting}
eststo est_a_clus : reg lsales lcapital llabor lmaterials, vce(cluster firm) \end{lstlisting}
\end{itemize}
    告訴Stata : 我知道有異質性變異數不符合OLS的假設,但將錯就錯,算出這時候該有的標準誤差吧,不過同一家廠商的變異數應該是一樣的。

\end{frame}

\begin{frame}
    \frametitle{OLS 不同標準誤差下的結果}

    \begin{table}
        {
\def\sym#1{\ifmmode^{#1}\else\(^{#1}\)\fi}
\begin{tabular}{l*{3}{c}}
\hline\hline
                    &\multicolumn{1}{c}{est\_a\_ols}&\multicolumn{1}{c}{est\_a\_r}&\multicolumn{1}{c}{est\_a\_clus}\\
\hline
log of capital      &       0.104\sym{***}&       0.104\sym{***}&       0.104\sym{***}\\
                    &   (0.00677)         &   (0.00786)         &    (0.0110)         \\
[1em]
log of labor        &       0.105\sym{***}&       0.105\sym{***}&       0.105\sym{***}\\
                    &   (0.00987)         &    (0.0105)         &    (0.0145)         \\
[1em]
log of materials    &       0.742\sym{***}&       0.742\sym{***}&       0.742\sym{***}\\
                    &   (0.00636)         &    (0.0103)         &    (0.0143)         \\
[1em]
Constant            &       1.641\sym{***}&       1.641\sym{***}&       1.641\sym{***}\\
                    &    (0.0485)         &    (0.0649)         &    (0.0916)         \\
\hline\hline
\multicolumn{4}{l}{\footnotesize Standard errors in parentheses}\\
\multicolumn{4}{l}{\footnotesize \sym{*} \(p<0.05\), \sym{**} \(p<0.01\), \sym{***} \(p<0.001\)}\\
\end{tabular}
}

    \end{table}

\end{frame}

\begin{frame}[fragile]
    \frametitle{固定效果}

    在估計固定效果之前,一定要記得告訴Stata 他是 Panel Date

    \begin{lstlisting}
xtset firm year \end{lstlisting}
    \vfill
    再估計固定效果 
\begin{lstlisting}
eststo est_e_fe : xtreg lsales lcapital llabor lmaterials, fe \end{lstlisting}

\end{frame}

\begin{frame}[fragile]
    \frametitle{隨機效果}
    不指定做法,預設上會採用隨機效果。但為了明確性,還是建議加上 re
\begin{lstlisting}
eststo est_d_re : xtreg lsales lcapital llabor lmaterials, re \end{lstlisting}

    \vfill
    可以接著用 Breusch and Pagan LM 檢定,來檢定隨機效果(相對於OLS)有沒有使用的必要
\begin{lstlisting}
xttest0 \end{lstlisting}  

\end{frame}

\begin{frame}[fragile]
    \frametitle{FE 與 RE,都幾?}

    用 Hausman Test
    \begin{lstlisting}
hausman est_e_fe est_d_re \end{lstlisting}  
    
    \vfill

    在 Hausman Test 中拒絕了虛無假設,因此認定 FE 與 RE 的係數顯著有差異。

    這時應該繼續用 FE 來避免不一致性。
    \vfill

    如果Hausman Test 無法拒絕虛無假設,表示其實 FE RE 兩著估計的係數差不多,
    那應該要選 RE,因為這種估計會更有效(efficient)。就像使用 FGSL 處理異質變異數一樣。

\end{frame}

\begin{frame}
    \frametitle{回歸結果}

    \begin{table}
        {
\def\sym#1{\ifmmode^{#1}\else\(^{#1}\)\fi}
\begin{tabular}{l*{4}{c}}
\hline\hline
                    &\multicolumn{1}{c}{est\_a\_ols}&\multicolumn{1}{c}{est\_d\_re}&\multicolumn{1}{c}{est\_e\_fe}&\multicolumn{1}{c}{est\_e\_fe\_cl}\\
\hline
log of capital      &       0.104\sym{***}&       0.102\sym{***}&      0.0519\sym{***}&      0.0519\sym{***}\\
                    &   (0.00677)         &   (0.00787)         &    (0.0130)         &    (0.0157)         \\
[1em]
log of labor        &       0.105\sym{***}&       0.130\sym{***}&       0.106\sym{***}&       0.106\sym{***}\\
                    &   (0.00987)         &    (0.0117)         &    (0.0205)         &    (0.0261)         \\
[1em]
log of materials    &       0.742\sym{***}&       0.700\sym{***}&       0.597\sym{***}&       0.597\sym{***}\\
                    &   (0.00636)         &   (0.00764)         &    (0.0117)         &    (0.0287)         \\
[1em]
Constant            &       1.641\sym{***}&       1.948\sym{***}&       3.500\sym{***}&       3.500\sym{***}\\
                    &    (0.0485)         &    (0.0638)         &     (0.166)         &     (0.290)         \\
\hline\hline
\multicolumn{5}{l}{\footnotesize Standard errors in parentheses}\\
\multicolumn{5}{l}{\footnotesize \sym{*} \(p<0.05\), \sym{**} \(p<0.01\), \sym{***} \(p<0.001\)}\\
\end{tabular}
}

    \end{table}

\end{frame}

\begin{frame}
    \frametitle{一點經濟學--- 固定規模報酬的檢定}

    Cobb-Douglas 生產函數 $Y=A K^{\alpha} L^{\beta} M^{\gamma}$
    \begin{itemize}
        \item 廠商理論與總體經濟常用的生產函數
        \item $\alpha + \beta + \gamma > 1$ --- 規模報酬遞增
        \item $\alpha + \beta + \gamma = 1$ --- 固定規模報酬 (CRTS)
    \end{itemize} 

    \begin{align*}
        \tilde{Y} &= A(2K)^{\alpha} (2L)^{\beta} (2M)^{\gamma} \\
        &= 2^{\alpha + \beta + \gamma} A K^{\alpha} L^{\beta} M^{\gamma}\\
        &= 2 Y \qquad \text{if } \alpha + \beta + \gamma = 1
    \end{align*}
    當所有要素都增加一倍,產出也剛好增加一倍,稱為固定規模報酬


\end{frame}

\begin{frame}
    \frametitle{對數下的 C-D 生產函數}
    將 Cobb-Douglas 生產函數取對數
    \begin{equation*}
        \ln(Y) = \ln(A) + \alpha \ln(K) + \beta \ln(L) + \gamma \ln(M)
    \end{equation*}

    對照

    \begin{align*}
        \ln(SALES_{it}) = \beta_1 &+ \beta_2 \ln(CAPITAL_{it})
        + \beta_3 \ln(LABOR_{it}) \\
        & + \beta_4 \ln(MATERIAL_{it}) 
        + u_i + e_{it}
    \end{align*}

\end{frame}

\begin{frame}[fragile]
    \frametitle{檢定固定規模報酬}

    題目 15-15 的 (b) 小題,用OLS以及穩健標準誤差下的結果,檢驗固定規模報酬

\begin{lstlisting}
global ols_list est_a_ols est_a_r est_a_clus

foreach m of global ols_list{
    est restore `m'
    test lcapital+llabor+lmaterials = 1
    lincom lcapital+llabor+lmaterials-1
}\end{lstlisting}

\end{frame}

\begin{frame}
    \frametitle{彈性}

    資本對產出的彈性定義為
    \begin{equation*}
        \frac{\ln(K)}{\ln(Y)}
    \end{equation*}

    恰好就是回歸式中 $CAPITAL$ 的係數。 
\end{frame}

\begin{frame}
    \frametitle{份額}
    而這些回歸係數還有一個經濟意涵,就是資本份額與勞動份額

    在完全競爭市場中, $MPL = w$,勞動邊際產出需要等於工資。

    Cobb-Douglas 稱產函數中的勞動邊際產出為

    \begin{equation*}
       MPL = \frac{dY}{dL} = \beta A K^{\alpha} L^{\beta - 1} M^{\gamma} = \beta \frac{Y}{L}
    \end{equation*}

    因此勞工的薪資則是

    \begin{equation*}
        wL = MPK \times L  = \beta \frac{Y}{L} \times L = \beta Y
    \end{equation*}

    也就是生產函數 $Y=A K^{\alpha} L^{\beta} M^{\gamma}$ 之下,勞工共分得總產出的 $\beta$ 部分。
    因此稱 $\beta$ 為勞動份額。

    一樣恰好就是回歸式中 $LABOR$ 的係數。
\end{frame}

\begin{frame}
    \frametitle{透過經濟理論尋找適合的計量模型}

    經濟理論 $\implies$ 變數關係 $\implies$ 計量模型 $\implies$ 實證研究

\end{frame}