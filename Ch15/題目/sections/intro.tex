\begin{frame}
    \frametitle{Panel Data}

    \begin{table}
        \begin{tabular}{c c | r r r}
            年份 & 工廠 & 產出 & 土地 & 勞動力 \\
            \hline \hline
            2020 & 甲 & 10箱 & 100坪 & 200人 \\
            2021 & 甲 & 15箱 & 96坪 & 300人 \\
            2022 & 甲 & 9箱 & 110坪 & 100人 \\
            \hline
            2020 & 乙 & 20箱 & 87坪 & 248人 \\
            2021 & 乙 & 29箱 & 93坪 & 310人 \\
            2022 & 乙 & 33箱 & 110坪 & 402人 \\
            \hline
            2020 & 丙 & 3箱 & 20坪 & 18人 \\
            2021 & 丙 & 6箱 & 28坪 & 32人 \\
            2022 & 丙 & 2箱 & 11坪 & 13人 \\
        \end{tabular}
    \end{table}
    \begin{itemize}
        \item 有時間
        \item 平衡(balanced)
    \end{itemize}
\end{frame}

\begin{frame}
    \begin{table}
        \begin{tabular}{c c | r r r}
            學生編號 & 老師 & 分數 & 曠課次數 & 家庭年收入 \\
            \hline \hline
            1 & 甲 & 90 & 10次 & 200萬 \\
            2 & 甲 & 87 & 9次 & 300萬 \\
            3 & 甲 & 80 & 11次 & 100萬 \\
            \hline
            4 & 乙 & 100 & 8次 & 248萬 \\
            5 & 乙 & 95 & 3次 & 310萬 \\
            6 & 乙 & 92 & 1次 & 402萬 \\
            \hline
            7 & 丙 & 88 & 2次 & 18萬 \\
            8 & 丙 & 57 & 8次 & 32萬 \\
            9 & 丙 & 76 & 1次 & 13萬 \\
        \end{tabular}
    \end{table}
    \begin{itemize}
        \item 沒有時間
    \end{itemize}
\end{frame}

\begin{frame}
    \frametitle{Panel Data Regression}
    當迴歸式中,要納入分群的「效果」,例如
    \begin{itemize}
        \item 不同廠商效率不同
        \item 不同老師教學品質不同
    \end{itemize}
    就可以被視為一個面板數據

    \begin{equation*}
        y_{it} = \beta_1 + \beta_2 x_{it} +  \beta_3 z_{it} + {\color{red} \alpha_i} + \epsilon_{it}
    \end{equation*}

    $\alpha_i$ 象徵這個人(廠商),在不同時間點,共同有的性質
    \begin{enumerate}
        \item 固定效果 FE --- $\alpha_i$ 代表不同個體有不同截距
        \item 隨機效果 RE --- $\alpha_i$ 是在誤差項
    \end{enumerate}

    \begin{block}{Panel Data 要有時間嗎?}
        一般來說講到 Panel Data 都會有時間,但沒有時間的資料,
        也可以當成 Panel Data 來處理 FE 與 RE,你們的作業就是這樣的例子
    \end{block}

\end{frame}

