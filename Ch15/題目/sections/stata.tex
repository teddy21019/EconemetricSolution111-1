\begin{frame}[fragile]
    \frametitle{指定為 Panel data}

    \begin{lstlisting}
xtset id time \end{lstlisting}
\begin{enumerate}
    \item \texttt{id} 替換成群組(廠商、教師、國家...)
    \item \texttt{time} 替代為時間變數
\end{enumerate}

如果不指定 time,則指會指定群組,依然可以做 FE RE,但沒辦法做「時間固定效果(time fixed effet)」
\end{frame}

\begin{frame}[fragile]
    \frametitle{固定效果}

    作法一、
    \begin{lstlisting}
reg y land labor i.firm \end{lstlisting}

    建議作法、
    \begin{lstlisting}
xtreg y land labor, fe \end{lstlisting}    

\end{frame}

\begin{frame}[fragile]
    \frametitle{隨機效果}

    \begin{lstlisting}
xtreg y land labor, re \end{lstlisting}    

    檢定隨機效果 (LM test)
    \begin{lstlisting}
xttest0 \end{lstlisting}    

\end{frame}

\begin{frame}
    \frametitle{FE 還是 RE?}
    如果兩者係數差不多,用 RE 比較有效 (efficient)。

    但如果差很多,那要用 FE,才會是一致的(consistent)

    \begin{block}{係數差不多?}
        用 Hausman Test 來檢定兩者回歸係數有無顯著差異

        H0 : 係數一樣\\
        Ha : 係數不一樣

        拒絕 Hausman Test 的虛無假設 --- 接受不一樣 $\implies$ 要用 FE
    \end{block}

\end{frame}

