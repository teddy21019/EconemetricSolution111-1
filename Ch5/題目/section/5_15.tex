\begin{frame}
    \frametitle{題幹}
    犯罪與懲罰間是什麼關係?Cornwell 與 Trumbull(1994)使用自北卡羅來納州的數據對此重要問題進行了研究。數據採用90個郡在1981 至1987年間的數據資料在資料檔crime中。

    使用1986年的數據,估計歸方程式,使用取對數後的犯罪率
     LCRMRTE 與
     逮捕機率 PRBARR (逮捕與犯罪的比率)、
     定罪機率PRBCONV(定罪與逮捕的比率)、
     監禁PRBPRIS(監禁與定罪的比率)、
     人均POLPC警察人數、
     以及WCON建的週工資。
     
     寫下你的發現,
    說明你期望每個變數對犯罪率有何影響,並注意估計係數是否有你所預期的符號,而檢定係數是否與零有很大的不同。哪些變數似乎對犯罪威懾最重要?
    你能解釋POLPC的符號嗎?
\end{frame}

\begin{frame}[fragile]
    \frametitle{回歸結果}
    \begin{table}
        \centering
        \scalebox{0.7}{
            {
\def\sym#1{\ifmmode^{#1}\else\(^{#1}\)\fi}
\begin{tabular}{l*{1}{c}}
\hline\hline
            &\multicolumn{1}{c}{(1)}\\
            &\multicolumn{1}{c}{lcrmrte}\\
\hline
prbarr      &      -2.044\sym{***}\\
            &     (0.265)         \\
[1em]
prbconv     &      -0.993\sym{***}\\
            &    (0.0908)         \\
[1em]
prbpris     &       0.300         \\
            &     (0.403)         \\
[1em]
polpc       &       150.2\sym{***}\\
            &     (13.53)         \\
[1em]
wcon        &     0.00211\sym{**} \\
            &  (0.000761)         \\
[1em]
\_cons      &      -3.364\sym{***}\\
            &     (0.317)         \\
\hline
\(N\)       &          90         \\
\(R^{2}\)   &       0.737         \\
\hline\hline
\multicolumn{2}{l}{\footnotesize Standard errors in parentheses}\\
\multicolumn{2}{l}{\footnotesize \sym{*} \(p<0.05\), \sym{**} \(p<0.01\), \sym{***} \(p<0.001\)}\\
\end{tabular}
}

        }
    \end{table}

\end{frame}

\begin{frame}[plain]
    這題很簡單,但是卻很重要


在進行回歸的時候,係數的解讀要很小心。
尤其在考慮因果關係時。
\pause
\par
這裡看的出來,警察越多,看似犯罪比率越高,好像會推出「警察會帶來犯罪」這樣的荒謬結論。
\pause
\par
然而警察會多,是因為犯罪率高,但是我們的模型並沒有考慮到這個可能性
這樣的問題稱作「內生性」。
\pause
\par
多變數回歸很方便,但除非是完全隨機的實驗(Randomized Experiments)
否則只要是觀察到的資料,都要考慮解釋變數是否有內生性問題。
\pause
\par
這也就是為什麼課本不會到這裡就結束了,經濟學家為了在不能做實驗的情況下
解釋效果發明了一整套的計量經濟作法,所以計量絕對不是丟給 stata 跑一個
reg y x1 x2 x3
就能拿諾貝爾獎的。

    

\end{frame}