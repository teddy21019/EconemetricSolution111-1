\begin{frame}
    \frametitle{題幹}

    使用資料檔 \texttt{toody5} 中的資料,估算模型
    \begin{equation}
        Y_t = \beta_1 + \beta_2 TREND_t + \beta_3 RAIN_t + \beta_4 RAIN_t^2 + \beta_5 (RAIN_t \times TREND_t) + e_t 
    \end{equation}

    \begin{itemize}
        \item $Y_t$:第t年小麥產量
        \item $TREND_t$:趨勢變數。1950=0, 1997=4.7 $(\frac{year - 1950}{10})$
        \item $RAIN$:總雨量
    \end{itemize}
\end{frame}

\begin{frame}
    \frametitle{a--估計值、標準誤差、t值、p值}
        \begin{enumerate}
            \item 先用 reg 跑回歸
            \item 用 esttab 顯示各項統計
        \end{enumerate}
\end{frame}

\begin{frame}[fragile]
    \frametitle{esttab 進階應用}
    \begin{lstlisting}
estout, cells("b(star fmt(3)) se t p") ///
    varlabels( c.rain#c.rain rain^2 c.rain#c.trend rain*trend _cons constant ) /// 重新命名 var
    starlevels(+ 0.1 * 0.05 ** 0.01 *** 0.001)\end{lstlisting}
    \vfill
    \begin{enumerate}
        \item \texttt{cells}:一個est中要放哪些統計量
        \item \texttt{varlabels}:重新命名變數顯示名稱
        \item \texttt{starlevels}:重新定義顯著水準符號
    \end{enumerate}
\end{frame}

\begin{frame}
    \frametitle{b -- 顯著性}
    透過 \texttt{starlevels}來重新定義顯著水準符號,觀察更多統計水準
\end{frame}

\begin{frame}
    \frametitle{c -- 是否符合預期?}
    這邊focus 在兩個交乘項的直覺
    \begin{itemize}
        \item $rain^2$ :雨太多反而會淹死,所以直覺上他的邊際產出會遞減,也就是預期
        \begin{center}
                $rain^2$ 係數 $<0$
        \end{center}
        \item $rain \times trend$::隨著技術進步,多下雨帶來的效益就會越小(因為這是抗旱的科技)
        反而在科技卓越時,下雨反而會帶來不便。因此預期
        \begin{center}
            交乘項係數 $<0$
        \end{center}
    \end{itemize}
\end{frame}

\begin{frame}
    \frametitle{d -- 估計邊際效果}
    想要看:
    \begin{enumerate}
        \item 1959年: $TREND_t=0.9$,$RAIN_t=2.98$
        \item 1995年: $TREND_t=4.5$,$RAIN_t=4.79$
    \end{enumerate}

    \vfill
    透過 \texttt{margin, dydx(rain) at(...)} 來看邊際影響
\end{frame}

\begin{frame}
    \frametitle{e -- d小題那兩年,使產出最大的降雨量}
    產出最大時,對降雨量微分應為零
    \begin{equation*}
        Y_t = \beta_1 + \beta_2 TREND_t + \beta_3 RAIN_t + \beta_4 RAIN_t^2 + \beta_5 (RAIN_t \times TREND_t) + e_t 
    \end{equation*}

    \begin{equation}
        \frac{d Y_t}{d RAIN_t} = \beta_3 + 2\beta_4 RAIN_t + \beta_5 TREND_t=0
    \end{equation}
    \begin{equation}
        \label{eq:max_rain}
        \{RAIN_t\}_{max} = \frac{-\beta_3 -\beta_5 TREND_t}{2\beta_4}
    \end{equation}

    \vfill
    透過 nlcom 輸入式~\ref{eq:max_rain}~找出這個值
\end{frame}