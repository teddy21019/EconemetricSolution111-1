\begin{frame}
    \frametitle{內生性}
    
    什麼時候(最需要)考慮內生性
    \begin{itemize}
        \item Y跟X被共同因子決定
        \begin{itemize}
            \item 薪資與工作決定--- 能力
            \item 嬰兒體重與母親抽菸 --- 健康意識
            \item 道路出事率與車種 --- 使用者族群
        \end{itemize}
        \item Y跟X同時被決定
        \begin{itemize}
            \item 價格與數量 --- 需求 \& 供給
        \end{itemize}
    \end{itemize}
\end{frame}

\begin{frame}[fragile]
    \frametitle{工具變數與 2SLS}
    解決方法
    \begin{enumerate}
        \item 找一個影響 X 但不影響 Y 的變數
        \item 做一次回歸把「乾淨的X」過濾出來
        \item 再一次回歸,把 Y 對 「乾淨的X」做回歸
    \end{enumerate}
    \vfill
    可以用 \texttt{ivregress} 指令輕鬆做到

    \begin{lstlisting}
        ivregress 2sls y x1 x2 (x3 = z1 z2 z3), first \end{lstlisting}
\begin{itemize}
    \item 解釋變數為 y
    \item \texttt{x3} 為內生變數,有些因子共同影響 \texttt{x3} 與 \texttt{y}
    \item \texttt{z1 z2 z3} 影響 \texttt{x3} 但不影響 \texttt{y}
\end{itemize}

\end{frame}


\begin{frame}
    \frametitle{聯立模型}
    如果模型長這樣
    \begin{align*}
        q_i &= \alpha_1 + \alpha_2 p_i + \alpha_3 A_i + \alpha_4 B_i + u_i &\text{供給} \\
        q_i &= \beta_1 + \beta_2 p_i + \beta_3 C_i + \beta_4 D_i + v_i &\text{需求} \\
    \end{align*}
    
    則 p, q 被共同決定,所以也內生性問題。
    \begin{enumerate}
        \item 先估計縮減式 $p_i = A+B+C+D$ 得到第一階段預測 $\hat{p}_i$
        \item 估計供給:
        \begin{equation*}
            q_i = \alpha_1 + \alpha_2\hat{p}_i + \alpha_3 A_i + \alpha_4 B_i + u_i
        \end{equation*}
        \item 估計需求:
        \begin{equation*}
            q_i = \beta_1 + \beta_2 \hat{p}_i + \beta_3 C_i + \beta_4 D_i + v_i
         \end{equation*}
    \end{enumerate}
\end{frame}

\begin{frame}[fragile]
    \frametitle{聯立模型 --- Stata}
    
    作法一、
    \begin{lstlisting}
ivregress 2sls q (p=C D) A B, first

ivregress 2sls q (p=A B) C D, first \end{lstlisting}
    \vfill
    作法二、用 3SLS 來估計聯立模型
    \begin{lstlisting}
reg3 (q p A B)(q p C D), endog(q p)\end{lstlisting}   

\end{frame}